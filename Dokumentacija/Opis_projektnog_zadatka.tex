\chapter{Opis projektnog zadatka}
		
		\textbf{\textit{dio 1. revizije}}\\
		
		\textit{Cilj ovog projekta je razviti online platformu za upravljanje i organizaciju kampa računarstva „Mlade nade“. Platforma će davati uvid u raspored, aktivnosti, te podatke o grupama i njenim članovima. Potencijalna korist ovog projekta jest jednostavnija organizacija kampa, te digitalizacija prijava na kamp, iz čega slijedi potencijalno veći broj prijavljenjih.}
		
		\textit{Sustav omogućava funkcionalnosti za 3 dionika: organizator, sudionik i animator. Neprijavljeni korisnici mogu vidjeti osnovne informacije o kampu, poput vremena održavanja, trajanja i aktivnosti. Za vrijeme prijava na kamp, na platformi će biti dostupne prijave na kamp za sudionike i/ili animatore. Animatori i sudionici prilikom prijave na kamp unose sljedeće podatke:}
		\begin{packed_item}
			\item \textit{ime i prezime}
			\item \textit{email adresa}
			\item \textit{broj telefona}
			\item \textit{datum i godina rođenja}
			\item \textit{motivacijsko pismo}
			\item \textit{broj odgovorne osobe (samo za sudionike mlađe od 18 godina)}
		\end{packed_item}
	
		\textit{Ako je nečija prijava prihvaćena, tu osobu se o tome obaviještava emailom te joj se šalju podatci potrebni za registraciju na platformu. Ako je nečija prijava odbijena, osoba se o tome također obavještava emailom. Korisnici zatim prilikom registracije za dobiveno korisničko ime (putem maila) upisuju lozinku.}
		
		\textit{Sudionici i animatori nakon prijave u sustav (prije početka kampa) vide samo odbrojavanje do početka kampa i imaju mogućnost kontaktiranja organizatora. Nakon početka kampa (za oba tipa korisnika) vodi na stranicu koja pokazuje njihov raspored ili agendu. Sudionici vide popis članova svoje grupe (u kojoj će sudjelovati na svim aktivnostima) i njihove kontakt podatke, kao i animatore s kojima će raditi i njihove kontakt podatke. Animatori vide popis svih grupa, njihovih članova i drugih animatora, kao i nihove kontakt podatke. Dodatno i sudionici i animatori vide popis aktivnosti na kojima su sudjelovali te imaju opciju ocjenjivanja aktivnosti te ostavljanja kratkog opisa njihovog dojma s aktivnošću. Nakon što je kamp završio, sudionici i animatori mogu ocjeniti i ostaviti vlastiti dojam za cjelokupno iskustvo. }
		
		\textit{Organizatori mogu uređivati osnovne informacije s početne stranice. Oni definiraju aktivnosti, te zadaju vrijeme prijava na kamp za sudionike i animatore. Također, vide popis prijava koje mogu prihvatiti ili odbiti. Organizatori određuju broj grupa u koje će, nakon završetka prijava na kamp, sustav slučajnim odabirom razvrstati sudionike, te ih naknadno mogu ručno razmještati. Organizatori mogu puniti raspored s aktivnostima, aktivnostima pridružiti animatore, te imaju popis svih povratnih ocjena po aktivnostima, koje mogu pretraživati po različitim atributima. Osim povratnih ocjena, organizatori mogu vidjeti popis svih aktivnosti, grupa, te registriranih korisnika, koje mogu uređivati. } 
		
		
		\eject
		
		\section{Primjeri u \LaTeX u}
		
		\textit{Ovo potpoglavlje izbrisati.}\\

		U nastavku se nalaze različiti primjeri kako koristiti osnovne funkcionalnosti \LaTeX a koje su potrebne za izradu dokumentacije. Za dodatnu pomoć obratiti se asistentu na projektu ili potražiti upute na sljedećim web sjedištima:
		\begin{itemize}
			\item Upute za izradu diplomskog rada u \LaTeX u - \url{https://www.fer.unizg.hr/_download/repository/LaTeX-upute.pdf}
			\item \LaTeX\ projekt - \url{https://www.latex-project.org/help/}
			\item StackExchange za Tex - \url{https://tex.stackexchange.com/}\\
		
		\end{itemize} 	


		
		\noindent \underbar{podcrtani tekst}, \textbf{podebljani tekst}, 	\textit{nagnuti tekst}\\
		\noindent \normalsize primjer \large primjer \Large primjer \LARGE {primjer} \huge {primjer} \Huge primjer \normalsize
				
		\begin{packed_item}
			
			\item  primjer
			\item  primjer
			\item  primjer
			\item[] \begin{packed_enum}
				\item primjer
				\item[] \begin{packed_enum}
					\item[1.a] primjer
					\item[b] primjer
				\end{packed_enum}
				\item primjer
			\end{packed_enum}
			
		\end{packed_item}
		
		\noindent primjer url-a: \url{https://www.fer.unizg.hr/predmet/proinz/projekt}
		
		\noindent posebni znakovi: \# \$ \% \& \{ \} \_ 
		$|$ $<$ $>$ 
		\^{} 
		\~{} 
		$\backslash$ 
		
		\begin{longtabu} to \textwidth {|X[8, l]|X[8, l]|X[16, l]|} %definicija širine tablice, širine stupaca i poravnanje
			
			%definicija naslova tablice
			\hline \multicolumn{3}{|c|}{\textbf{naslov unutar tablice}}	 \\[3pt] \hline
			\endfirsthead
			
			%definicija naslova tablice prilikom prijeloma
			\hline \multicolumn{3}{|c|}{\textbf{naslov unutar tablice}}	 \\[3pt] \hline
			\endhead
			
			\hline 
			\endlastfoot
			
			\rowcolor{LightGreen}IDKorisnik & INT	&  	Lorem ipsum dolor sit amet, consectetur adipiscing elit, sed do eiusmod  	\\ \hline
			korisnickoIme	& VARCHAR &   	\\ \hline 
			email & VARCHAR &   \\ \hline 
			ime & VARCHAR	&  		\\ \hline 
			\cellcolor{LightBlue} primjer	& VARCHAR &   	\\ \hline 
			
		\end{longtabu}
		

		\begin{table}[H]
			
			\begin{longtabu} to \textwidth {|X[8, l]|X[8, l]|X[16, l]|} 
				
				\hline 
				\endfirsthead
				
				\hline 
				\endhead
				
				\hline 
				\endlastfoot
				
				\rowcolor{LightGreen}IDKorisnik & INT	&  	Lorem ipsum dolor sit amet, consectetur adipiscing elit, sed do eiusmod  	\\ \hline
				korisnickoIme	& VARCHAR &   	\\ \hline 
				email & VARCHAR &   \\ \hline 
				ime & VARCHAR	&  		\\ \hline 
				\cellcolor{LightBlue} primjer	& VARCHAR &   	\\ \hline 
				
				
			\end{longtabu}
	
			\caption{\label{tab:referencatablica} Naslov ispod tablice.}
		\end{table}
		
		
		%unos slike
		\begin{figure}[H]
			\includegraphics[scale=0.4]{slike/aktivnost.PNG} %veličina slike u odnosu na originalnu datoteku i pozicija slike
			\centering
			\caption{Primjer slike s potpisom}
			\label{fig:promjene}
		\end{figure}
		
		\begin{figure}[H]
			\includegraphics[width=.9\linewidth]{slike/aktivnost.PNG} %veličina u odnosu na širinu linije
			\caption{Primjer slike s potpisom 2}
			\label{fig:promjene2} %label mora biti drugaciji za svaku sliku
		\end{figure}
		
		Referenciranje slike \ref{fig:promjene2} u tekstu.
		
		\eject
		
	