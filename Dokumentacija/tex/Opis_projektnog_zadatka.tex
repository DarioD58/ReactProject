\chapter{Opis projektnog zadatka}
			
		Cilj ovog projekta je razviti online platformu za upravljanje i organizaciju kampa računarstva „Mlade nade“. Platforma će davati uvid u raspored, aktivnosti, te podatke o grupama i njenim članovima. Potencijalna korist ovog projekta jest jednostavnija organizacija kampa, te digitalizacija prijava na kamp, iz čega slijedi potencijalno veći broj prijavljenjih.
		
		Sustav omogućava funkcionalnosti za 3 dionika: organizator, sudionik i animator. Neprijavljeni korisnici mogu vidjeti osnovne informacije o kampu, poput vremena održavanja, trajanja i aktivnosti. Za vrijeme prijava na kamp, na platformi će biti dostupne prijave na kamp za sudionike i/ili animatore. Animatori i sudionici prilikom prijave na kamp unose sljedeće podatke:
		\begin{packed_item}
			\item ime i prezime
			\item email adresa
			\item broj telefona
			\item datum i godina rođenja
			\item motivacijsko pismo
			\item broj odgovorne osobe (samo za sudionike mlađe od 18 godina)
		\end{packed_item}
	
		Ako je nečija prijava prihvaćena, tu osobu se o tome obavještava emailom te joj se šalju podatci potrebni za registraciju na platformu. Ako je nečija prijava odbijena, osoba se o tome također obavještava emailom. Korisnici zatim prilikom registracije za dobiveno korisničko ime (putem maila) upisuju lozinku.
		
		Sudionici i animatori nakon prijave u sustav (prije početka kampa) vide samo odbrojavanje do početka kampa i imaju mogućnost kontaktiranja organizatora. Nakon početka kampa (za oba tipa korisnika) vodi na stranicu koja pokazuje njihov raspored. Sudionici vide popis članova svoje grupe (u kojoj će sudjelovati na svim aktivnostima) i njihove kontakt podatke, kao i animatore s kojima će raditi i njihove kontakt podatke. Animatori vide popis svih grupa, njihovih članova i drugih animatora, kao i njihove kontakt podatke. Dodatno i sudionici i animatori vide popis aktivnosti na kojima su sudjelovali te imaju opciju ocjenjivanja aktivnosti te ostavljanja kratkog opisa njihovog dojma s aktivnošću. Nakon što je kamp završio, sudionici i animatori mogu ocijeniti i ostaviti vlastiti dojam za cjelokupno iskustvo. 
		
		Organizatori mogu uređivati osnovne informacije s početne stranice. Oni definiraju aktivnosti, te zadaju vrijeme prijava na kamp za sudionike i animatore. Također, vide popis prijava koje mogu prihvatiti ili odbiti. Organizatori određuju broj grupa u koje će, nakon završetka prijava na kamp, sustav slučajnim odabirom razvrstati sudionike, te ih naknadno mogu ručno razmještati. Organizatori mogu puniti raspored s aktivnostima, aktivnostima pridružiti animatore, te imaju popis svih povratnih ocjena po aktivnostima, koje mogu pretraživati po različitim atributima. Osim povratnih ocjena, organizatori mogu vidjeti popis svih aktivnosti, grupa, te registriranih korisnika, koje mogu uređivati.  
		
		
		\eject
		
		
		
	