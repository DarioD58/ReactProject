\chapter{Specifikacija programske potpore}
		
	\section{Funkcionalni zahtjevi}
			
			\textbf{\textit{dio 1. revizije}}\\
			
			\textit{Navesti \textbf{dionike} koji imaju \textbf{interes u ovom sustavu} ili  \textbf{su nositelji odgovornosti}. To su prije svega korisnici, ali i administratori sustava, naručitelji, razvojni tim.}\\
				
			\textit{Navesti \textbf{aktore} koji izravno \textbf{koriste} ili \textbf{komuniciraju sa sustavom}. Oni mogu imati inicijatorsku ulogu, tj. započinju određene procese u sustavu ili samo sudioničku ulogu, tj. obavljaju određeni posao. Za svakog aktora navesti funkcionalne zahtjeve koji se na njega odnose.}\\
			
			
			\noindent \textbf{Dionici:}
			
			\begin{packed_enum}
				
				\item Sudionik
				\item Animator				
				\item Organizator
				
			\end{packed_enum}
			
			\noindent \textbf{Aktori i njihovi funkcionalni zahtjevi:}
			
			
			\begin{packed_enum}
				\item  \underbar{Neregistrirani/neprijavljeni korisnik (inicijator) može:}
				
				\begin{packed_enum}
					
					\item pregledati osnovne informacije o kampu
					\item prijaviti se na kamp (u kapacitetu sudionika ili animatora)
					\item  registrirati se u sustav pomoću podataka dobivenih mailom
					\item prijaviti se u sustav s korisničkim imenom i lozinkom
				\end{packed_enum}
			
				\item  \underbar{Korisnik (sudionik, organizator, animator)(inicijator) može:}
				
				\begin{packed_enum}
					
					\item pregledavati i mijenjati svoje korisničke podatke
					\item izbrisati svoj korisnički račun
					
				\end{packed_enum}
					\item  \underbar{Sudionik, animator (inicijator) može:}
				\begin{packed_enum}
					
					\item vidjeti odbrojavanje do početka kampa (prije početka kampa)
					\item kontaktirati organizatora (prije početka kampa)
					\item pregledati svoj raspored aktivnosti i agendu
					\item vidjeti popis aktivnosti na kojima je sudjelovao te ih ocjeniti i ostaviti komentar
					\item uređivati ostavljenu ocjenu i komentar
					\item po završetku kampa ocjeniti i ostaviti komentar za cjelokupno iskustvo
					
				\end{packed_enum}
					\item  \underbar{Sudionik (inicijator) može:}
				\begin{packed_enum}
					
				    \item vidjeti svoju grupu, popis njenih članova, i njihove kontakt podatke, kao i animatora s kojima će raditi/je radio te njihove kontakt podatke
					
				\end{packed_enum}
					\item  \underbar{Animator (inicijator) može:}
				\begin{packed_enum}
					
					\item vidjeti popis svih grupa, njihovih članova i drugih animatora, kao i njihove kontakt podatke
				
				\end{packed_enum}
					\item  \underbar{Baza podataka (sudionik) može:}
				\begin{packed_enum}
					
					\item pohraniti sve podatke o korisnicima, aktivnostima, grupama i ocjenjivanju
					
				\end{packed_enum}
				\item  \underbar{Sustav (sudionik) može:}
		    	\begin{packed_enum}
				
				\item pristupiti bazi podataka na zahtjev korisnika sustava
				\item poslati mail sa podatcima za registraciju sudionicima i animatorima 
			    \end{packed_enum}
		    		\item  \underbar{Organizator (inicijator) može:}
		    	\begin{packed_enum}
		    		
		    		\item vidjeti popis svih registriranih korisnika, aktivnosti, grupa i ocjenjivanja, uređivati te popise i brisati stavke tih popisa
		    		\item uređivati osnovne informacije početne stranice
		    		\item definirati aktivnosti te im pridruživati animatore
		    		\item zadati vrijeme prijava na kamp za sudionike i/ili animatore
		    		\item vidjeti popis prijava, te ih prihvatiti ili odbiti
		    		\item odrediti broj grupa na kampu
		    		\item premjestiti sudionika u drugu grupu
		    		\item  puniti raspored s aktivnostima, uređivati ga i brisati aktivnosti iz njega
		    		
		    	\end{packed_enum}
			
			\end{packed_enum}
			
			\eject 
			
			
				
			\subsection{Obrasci uporabe}
				
				\textbf{\textit{dio 1. revizije}}
				
				\subsubsection{Opis obrazaca uporabe}
					\textit{Funkcionalne zahtjeve razraditi u obliku obrazaca uporabe. Svaki obrazac je potrebno razraditi prema donjem predlošku. Ukoliko u nekom koraku može doći do odstupanja, potrebno je to odstupanje opisati i po mogućnosti ponuditi rješenje kojim bi se tijek obrasca vratio na osnovni tijek.}\\
					

					\noindent \underbar{\textbf{UC$<$broj obrasca$>$ -$<$ime obrasca$>$}}
					\begin{packed_item}
	
						\item \textbf{Glavni sudionik: }$<$sudionik$>$
						\item  \textbf{Cilj:} $<$cilj$>$
						\item  \textbf{Sudionici:} $<$sudionici$>$
						\item  \textbf{Preduvjet:} $<$preduvjet$>$
						\item  \textbf{Opis osnovnog tijeka:}
						
						\item[] \begin{packed_enum}
	
							\item $<$opis korak jedan$>$
							\item $<$opis korak dva$>$
							\item $<$opis korak tri$>$
							\item $<$opis korak četiri$>$
							\item $<$opis korak pet$>$
						\end{packed_enum}
						
						\item  \textbf{Opis mogućih odstupanja:}
						
						\item[] \begin{packed_item}
	
							\item[2.a] $<$opis mogućeg scenarija odstupanja u koraku 2$>$
							\item[] \begin{packed_enum}
								
								\item $<$opis rješenja mogućeg scenarija korak 1$>$
								\item $<$opis rješenja mogućeg scenarija korak 2$>$
								
							\end{packed_enum}
							\item[2.b] $<$opis mogućeg scenarija odstupanja u koraku 2$>$
							\item[3.a] $<$opis mogućeg scenarija odstupanja  u koraku 3$>$
							
						\end{packed_item}
					\end{packed_item}
				
					
				\subsubsection{Dijagrami obrazaca uporabe}
					
					\textit{Prikazati odnos aktora i obrazaca uporabe odgovarajućim UML dijagramom. Nije nužno nacrtati sve na jednom dijagramu. Modelirati po razinama apstrakcije i skupovima srodnih funkcionalnosti.}
				\eject		
				
			\subsection{Sekvencijski dijagrami}
				
				\textbf{\textit{dio 1. revizije}}\\
				
				\textit{Nacrtati sekvencijske dijagrame koji modeliraju najvažnije dijelove sustava (max. 4 dijagrama). Ukoliko postoji nedoumica oko odabira, razjasniti s asistentom. Uz svaki dijagram napisati detaljni opis dijagrama.}
				\eject
	
		\section{Ostali zahtjevi}
		
			\textbf{\textit{dio 1. revizije}}\\
		 
			 \textit{Nefunkcionalni zahtjevi i zahtjevi domene primjene dopunjuju funkcionalne zahtjeve. Oni opisuju \textbf{kako se sustav treba ponašati} i koja \textbf{ograničenja} treba poštivati (performanse, korisničko iskustvo, pouzdanost, standardi kvalitete, sigurnost...). Primjeri takvih zahtjeva u Vašem projektu mogu biti: podržani jezici korisničkog sučelja, vrijeme odziva, najveći mogući podržani broj korisnika, podržane web/mobilne platforme, razina zaštite (protokoli komunikacije, kriptiranje...)... Svaki takav zahtjev potrebno je navesti u jednoj ili dvije rečenice.}
			 
			 
			 
	